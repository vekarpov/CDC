\documentclass[a4paper]{article}
\usepackage[14pt]{extsizes}
\usepackage[
    left=20mm, top=15mm, right=15mm, bottom=15mm,
    nohead, footskip=10mm
]{geometry}

\usepackage[russian]{babel}
\usepackage[T2A]{fontenc}
\usepackage{fontspec}
\setmainfont{FreeSerif}
\usepackage{graphicx}
\usepackage{amsthm}
\usepackage{amsmath}

\newtheorem{theorem}{Теорема}[section]
\newtheorem{corollary}{Corollary}[theorem]
\newtheorem{lemma}[theorem]{Лемма}

\theoremstyle{definition}
\newtheorem{definition}{Определение}[section]

\usepackage{setspace,amsmath}

\begin{document}

\begin{center}
    \textbf{ПРАВИТЕЛЬСТВО РОССИЙСКОЙ ФЕДЕРАЦИИ} \\
    \textbf{НАЦИОНАЛЬНЫЙ ИССЛЕДОВАТЕЛЬСКИЙ УНИВЕРСИТЕТ} \\
    \textbf{«ВЫСШАЯ ШКОЛА ЭКОНОМИКИ»} \\
    \hfill\break
    \normalsize{Факультет компьютерных наук} \\
    Образовательная программа «Прикладная математика и информатика» \\
    \vspace{4.2cm}
    \textbf{Отчет об исследовательском проекте} \\
    \hfill\break
    на тему "Пространства циклов в графах" \\
    \hfill\break
    (промежуточный, этап 2)
\end{center}
\vspace{3cm}
\textbf{Выполнил:} \\
Студент группы БПМИ181 \hfill \underline{\hspace{5cm}} \hfill В.Е. Карпов \\
\vspace{0.1cm} \\
\hspace*{\fill} Дата \;\underline{\hspace{4cm}}\; 2020 \\
\vspace{0.1cm} \\
\textbf{Принял:} \\
Руководитель проекта \hfill Вялый Михаил Николаевич
\begin{flushright}
    Профессор
\end{flushright}
\vspace{0.5cm}
Дата \;\underline{\hspace{4cm}}\; 2020 \hfill
Оценка \quad \underline{\hspace{1.5cm}} \hfill
\underline{\hspace{5cm}} \\
\vfill
\begin{center}
    \textbf{Москва 2020}
\end{center}
\thispagestyle{empty}

\newpage

\tableofcontents

\newpage

\section{Введение}

\subsection{Объект и предмет исследования}

Рассмотрим множество графов, с $n$ рёбрами, возможно, содержащиx кратные рёбра, но не cодержащиx петель. Тогда для каждого из них можно рассмотреть пространство его циклов. В данном случае не требуется простота циклов, единственное ограничение в том, что цикл не должен проходить два раза по одному ребру. Легко увидеть, что это пространство будет замкнуто относительно операции сложения по модулю 2.

Будем обозначать такое пространство для графа $G$ как $Cyc(G)$.

\begin{definition}
	Будем обозначать множество всех подмножеств из $n$ рёбер за $V_n$
\end{definition}



\begin{definition} \label{d1}
	Множество $M \subseteq V_n$ назовём разрешённым, если существует граф $G$ из $n$ рёбер, что $Cyc(G) \supseteq M$, где $Cyc(G)$ - пространство циклов графа $G$.
\end{definition}

Таким образом, на множестве $V$ можно ввести функцию $f$

\begin{equation*}
f(x) = 
\begin{cases}
0, &\text{x разрешённое}\\
1, &\text{иначе}
\end{cases}
\end{equation*}

Для того, чтобы сформулировать цель работы, докажем вспомогательные теорему и введём определения:

\begin{lemma}\label{l1}
	Если для двух множеств $A$, $B \subseteq V_n$, $A \supseteq B$, то $f(A) \ge f(B)$
\end{lemma}

\begin{proof}
	В силу того, что функция $f$ может принимать только два значения, лемма может быть неверна лишь в случае $f(A) = 0, f(B) = 1$. То есть, по определению \ref{d1}, существует граф $G$, для которого $A \subseteq Cyc(G) \Rightarrow B \subseteq A  \subseteq Cyc(G) \Rightarrow f(B) = 1$. Противоречие. 
\end{proof}

Значит, функция $f$ не убывает на частично упорядоченном множестве $V$.

\begin{definition}
	Верхним нулём для функции $f$ и множества $V$ будем называть такое $A$, что $f(A) = 0$ и $(B \supseteq A ) \Rightarrow f(B) = 1$
\end{definition}
\begin{definition}
	Нижней единицей для функции $f$ и множества $V$ будем называть такое $A$, что $f(A) = 1$ и $(B \subseteq A ) \Rightarrow f(B) = 0$
\end{definition}

Здесь и далее будем считать, что было зафиксировано некоторое $n$, и теперь множества верхних нулей и нижних единиц всегда рассматриваются на $V_n$.

\begin{theorem}
	Для $X \in V_n, f(X) = 1$ тогда и только тогда, когда существует нижняя единица, которая вложена в $X$. Аналогично, $f(X) = 0$ тогда и только тогда когда существует верхний ноль, в который вложен $X$.
\end{theorem}
\begin{enumerate}
	\item Нижние единицы
	
	$(\Rightarrow)$ Пусть $Y$ -- та нижняя единица, в которую вложен $X$. Тогда, из леммы \ref{l1} $X \supseteq Y \Rightarrow f(X) \ge f(Y) = 1 \Rightarrow f(X) = 1$
	
	$(\Leftarrow)$ Пусть $f(B) = 1$. Рассмотрим минимальное по вложению $C$, что $C \supseteq X, f(C) = 1$. Такое будет, т.к. число рёбер конечно. А если $C$ минимально по вложению, то, значит, функция от любого его подмножества равна $0$, значит, $C$ принадлежит множеству нижних единиц.
	
	\item Верхние нули
	
	$(\Rightarrow)$ Пусть $Y$ -- верхний ноль, вложенный в $X$. Тогда, из леммы \ref{l1} $Y \supseteq X \Rightarrow f(X) \le f(Y) = 0 \Rightarrow f(X) = 0$
	
	$(\Leftarrow)$ Пусть $f(B) = 0$. Рассмотрим максимальное по вложению $C$, что $C \subseteq X, f(C) = 0$. Такое будет, т.к. число рёбер конечно. А если $C$ максимально по вложению, то, значит, функция от любого множества, включающего его равна $1$, значит, $C$ принадлежит множеству верхних нулей.
\end{enumerate}

Таким образом, знание множества верхних нулей или нижних единиц позволяет легко определять значение функции $f$ для произвольного подмножества $V_n$.

Основной целью исследовательского проекта является исследование множества нижних единиц и верхних нулей на $V_n$.




\newpage
\section{Сведение к линейной алгебре}

Оперировать с множеством циклов не очень удобно, потому что оно дважды экспоненциально растёт от числа рёбер. Поэтому, рассмотрим его как пространство циклов с операцией сложения по модулю 2.

Тогда, из этого пространства можно выделить базис и оперировать с его размерностью. Выделим особый базис из пространства циклов графа $G$.

Рассмотрим какое-нибудь остовное дерево для $G$ (если граф несвязный, то рассмотрим каждую компоненту связности по-отдельности). Тогда любое ребро, не вошедшее в остов при добавлении к остову будет образовывать ровно один цикл.

\begin{theorem} \label{t2}
	Объединение всех циклов, полученных таким путём будет являться базисом для пространства циклов графа.
\end{theorem}
\begin{proof}
	Для начала, независимость. Если $e$ - номер ребра, добавлением которого был получен какой-то цикл из базиса, то ребра $e$ нет ни в каком другом цикле базиса, а, значит, этот цикл не может быть получен как сумма каких-то других циклов из базиса.
	
	Теперь докажем, что любой другой цикл можно получить таким образом. Обозначим множество рёбер, не вошедших в остов за $E$. Тогда пусть $E'$ - это пересечение $E$ и цикла, который хочется получить. Т.к. каждое из рёбер $E'$ присутствует только в одном цикле из базиса, то рассмотрим сумму всех циклов из базиса, в которых присутствуют рёбра из $E'$. Пусть у нас получился не тот цикл, который надо было получить. Тогда рассмотрим симметрическую разность этих циклов -- это тоже будет цикл, но т.к. все рёбра из $E$ или одновременно присутствуют или одновременно отсутствуют и в том который надо было получить, и в полученном, то в полученной симметрической разности не будет рёбер из $E$. Значит, симметрическая разность -- цикл на остовном дереве, значит, он не включает в себя ни одно ребро, что означает, что всё-таки цикл, полученный из базиса равен начальному.
\end{proof}

Из этого важно сделать следствие о том, что размерность пространства циклов для связного графа на $n$ вершинах c $m$ рёбрами всегда равна $m - (n - 1)$

\section{Актуальность исследования}

Задача $CDC$ (Circuit Double Cover) является одной из нерешённых задач в теории графов.

\begin{theorem}
	[CDC] У любого графа, не содержащего мостов, существует такое множество циклов, что каждое ребро появляется в нём ровно два раза.
\end{theorem}

Пусть найдётся такое подпространство, не принадлежащее циклическим, и не содержащее единичных векторов, которое принадлежит пространству разрезов графа (то есть ортогональное дополнение которого принадлежит пространству циклов). Тогда понятно, что граф, пространство разрезов которого оно содержит, будет контрпримером к CDC.

Таким образом, зная структуру нижних единиц, можно будет попытаться найти контрпример к $CDC$.

\section{Верхние нули}

\begin{definition}
	Назовём усиленным полным графом на $n$ вершинах такой граф без петель на $n$ вершинах, из которого можно выкинуть какое-то число рёбер так, чтобы остался полный граф на $n$ вершинах.
\end{definition}

Докажем, что \\$\{ Cyc(G) | G - {\text{усиленный полный граф c $m$ рёбрами c числом вершин не равным 4}} \}$ и есть множество верхних нулей для $V_m$.

\begin{lemma}
	Для любого верхнего нуля существует граф, пространство циклов которого в точности равно верхнему нулю.
\end{lemma}
\begin{proof}
	Пусть для какого-то верхнего нуля $A$ это не так. Т.к. функция от него равна нулю, то существует граф $G$, в пространство циклов которого вложен этот верхний ноль. $Cyc(G) \supseteq A, f(Cyc(G)) = 0 \Rightarrow A - $ не верхний ноль.
\end{proof}

\begin{lemma}
	Граф, не являющийся усиленным полным, не является верхним нулём.
\end{lemma}
\begin{proof}
	Граф не усиленный полный, значит существуют две вершины, не соединённые ребром. Рассмотрим граф, в котором эти две вершины объединены в одну. Легко заметить, что пространство циклов от такой операции не уменьшилось. Если бы оно осталось таким же, то начальный граф не был бы нулём из условия минимальности по числу вершин в определении \ref{d2}. Если же оно увеличилось, то соответствующее графу множество не является верхним нулём, т.к. функция от пространства циклов нового графа равна нулю, и пространство циклов нового графа включает в себя множеcтво, соответствующее старому.
\end{proof}

Теперь можно говорить, что верхними нулями могут являться лишь усиленные полные графы.
Тогда для каждого верхнего нуля рассмотрим минимальный по числу вершин граф, пространство циклов которого равно в точности верхнему нулю.

Докажем, что это определение корректно, то есть не бывает двух графов из одного числа вершин с точностью до изоморфизма, у которых пространства циклов совпадают.
\begin{theorem} \label{r1}
	Если у двух усиленных полных графов пространства циклов совпадают, то их рёберные графы тоже совпадают
\end{theorem}
\begin{proof}
	Для двух рёбер бывают три ситуации: не пересекаются, пересекаются и кратны. Докажем, что кратность в одном графе влечёт кратность в другом, а пересечение в одном влечёт пересечение в другом для любых двух рёбер.
	
	(1) Пусть два ребра кратны, тогда есть цикл из них двоих, значит, они кратны и в другом графе.
	
	(2) Пусть два ребра пересекаются. Т.к. рассматриваем только усиленные полные графы, то, значит, есть цикл из 3х, включающий эти два (если они, конечно, не кратны). Значит, и во втором есть этот цикл, а, значит и во втором рёбра пересекаются
\end{proof}

Теперь по Теореме Уитни об изоморфизме графов можно сказать, что если пространства циклов равны, то графы изоморфны.
(На самом деле, теорема Уитни утверждает, что если пространства циклов равны, то графы изоморфны для всех случаев, кроме пары графов $K_3$, $K_{1, 3}$, но второй из них не является полным, а, значит, не может встретиться в процессе доказательства)

\begin{definition} \label{d2}
	Будем называть граф верхним нулём, если он является минимальным по числу вершин графом, пространство циклов которого равно в точности верхнему нулю для какого-то верхнего нуля.
\end{definition}


Вообще, полный усиленный граф не будет являться верхним нулём т.т.т.к. будет существовать граф, пространство циклов которого будет включать в себя пространство циклов начального. Вспоминая про следствие из теоремы \ref{t2} можно сразу сказать, что это не может быть граф с большим числом вершин т.к. размерность пространства циклов для него будет меньше размерности пространства циклов начального, и, значит, циклов в нём будет банально меньше.
Значит, если такой граф и существует, то число вершин в нём не превосходит числа вершин в начальном.

Докажем вспомогательную лемму:
\begin{lemma}
	Для любого числа вершин $n$ усиленный полный граф $A$ на $n$ вершинах с $m_1$ рёбрами и усиленный полный граф $B$ на $n$ вершинах, что $B \supseteq A$ с $m_2$ рёбрами или являются или не являются верхними нулями одновременно на $V_{m_1}$ и $V_{m_2}$ соответственно.
\end{lemma}
\begin{proof}
	Пусть $A$ не является верхним нулём. Значит, существует граф с не большим числом вершин, чем у данного, пространство циклов которого включает в себя $Cyc(A)$. Назовём его $C$.
	Построим базис $B$ на каком-нибудь остовном дереве, полностью лежащем в $A$. Тогда все добавленные в $B$ рёбра относительно $A$ превратятся в циклы длины $2$, такие, что одно из рёбер в этих циклах из $A$.  Пусть $x$ -- ребро, которое есть в $B$, но нет в $A$. Пусть в базисе оно находится в паре с ребром $y$. Построим $C'$, добавив для каждого такого $x$ в $C$ ребро кратное $y_x$.
	
	Построим базис $C'$ как базис $C$ плюс $(x, y_x)$ для всех $x$. Понятно, что это будет базис. С другой стороны, т.к. базис $C$ можно предcтавить, как базис $A$ плюс ещё что-то, базис $C'$ можно представить как базис $B$ и ещё что-то ненулевое. Значит, $B$ - тоже не верхний ноль.
	
	Теперь пусть $B$ не является верхним нулём. Так же рассмотрим граф $C$. Заметим, что если $x$ и $y$ - кратные рёбра в $B$, то они должны оставаться кратными в $C$ (т.к. должен сохраняться цикл $(x, y)$). Значит, при удалении $y$ из $C$ удаляться лишь циклы, содержащие $y$. А т.к. все рёбра, добавленные в $B$ относительно $A$, кратные для каких-то рёбер из $A$, то удалив из $C$ все эти рёбра, получим граф, доказывающий, что $A$ не верхний ноль.
\end{proof}

\begin{theorem}
Никакой граф на четырёх вершинах не является верхним нулём
\end{theorem}
\begin{proof}
Для начала докажем, что полный граф на 4х вершинах не является верхним нулём.\\
\includegraphics{project.png}\\
На рисунке изображён граф на трёх вершинах, пространство циклов которого содержит пространство циклов полного на четырёх вершинах.

Из предыдущей леммы, т.к. все усиленные полные на четырёх вершинах содержат обычный полный, то все усиленные полные на 4х вершинах также не являются верхними нулями.
\end{proof}
Отдельно рассмотрим случай усиленного полного графа на трёх вершинах - в нём есть цикл на три вершины, а любой граф на двух вершинах не содержит нечётного цикла.
\begin{theorem}
Если в усиленном полном графе $A$ более 4x вершин, то пространство циклов никакого графа, содержащего меньшее чем $A$ число вершин не содержит $Cyc(A)$.
\end{theorem}
\begin{proof}
Рассмотрим базис пространства циклов для $A$, где в качестве остова выбраны только рёбра выходящие из одной вершины. Тогда любой цикл из базиса будет состоять из двух или трёх рёбер, в котором одно или два соответственно ребра будут из остова.

Пусть существует граф $B$, в котором меньше вершин и пространство циклов которого включает в себя $Cyc(A)$. Для удобства введём биекцию $g$, которая переводит ребро из $A$  в соответствующее ему ребро из $B$. Рассмотрим любые два ребра $x$ и $y$ из взятого остова $A$. Т.к. для них в $A$ существует ребро $c$, что $(a, b, c)$ - цикл, то $(g(a), g(b), g(c))$ - тоже цикл. Значит, $g(a)$ и $g(b)$ имеют общую вершину. Таким образом, или $g(a_1), g(a_2) \ldots g(n - 1)$ перешли в рёбра между какими-то тремя вершинами, или перешли в рёбра, которые все исходят из одной вершины. Т.к. доказываем для $n \ge 5 \Rightarrow (n - 1) \ge 4$, то, по принципу Дирихле, какие-то два ребра перейдут в кратные, но, как уже было сказано выше, для любых двух рёбер из остова существует цикл на три ребра, проходящий через них, а его в $B$ явно не будет.
\end{proof}

Теперь осталось сказать про графы с одинаковым числом вершин. В силу того, что их размерности равны, вложение превращается в равенство, а этот случай уже был доказан в теореме \ref{r1}.

Таким образом, действительно, $\{ Cyc(G) | G -\\ {\text{усиленный полный граф c $m$ рёбрами c числом вершин не равным 4}} \}$ и есть множество верхних нулей для $V_m$.

Таким образом, структура верхних нулей легко описывается


\section{Метод редукции столбцов}

\begin{definition}
	Назовём матрицу $A$ разрешенной, если $IA$ вкладывается в циклическое пространство некоторого графа (здесь $I$ - единичная матрица)
\end{definition}
Другими словами, хочется реализовать строки $A$ как такие подмножества рёбер некоторого графа, что в каждом графе, состоящем из рёбер каждого подмножества, будет ровно две вершины нечётной степени (в этом случае, если ребро из $I$ соединит эти две вершины, получится цикл)

Несколько простых вспомогательных утверждений про разрешённые матрицы:

\begin{theorem}
	Выбрасывание строки с одной единицей не изменяет разрешённости матрицы.
\end{theorem}
\begin{proof}
	Для любого распределения рёбер графа между столбцами, в графе, состоящем только из подмножества рёбер строки с одной единицей, будет ровно одно ребро, а, значит, в нём будет ровно две вершины нечётной степени. То есть, добавление или удаление такой строки не влияет на разрешённость матрицы.
\end{proof}

\begin{theorem}
	Замена одинаковых строк на одну копию не меняет разрешённость матрицы.
\end{theorem}
\begin{proof}
	Для любой повторяющихся реализации строк, как подмножества рёбер некого графа, свойство будет выполняться или не выполняться для всех одинаковых строк одновременно. А так как требуется выполнение свойства для всех строк, то можно заменить несколько строк на одну.
\end{proof}

\begin{theorem}
	Добавление столбца нулей не меняет разрешённость матрицы.
\end{theorem}
\begin{proof}
	Следует прямо из определения
\end{proof}

\begin{definition}
	Назовём редукцией $n$ столбцов матрицы $M$ операцию, при которой все столбцы, кроме первого, удаляются из матрицы, $IM$ а столбец $a_1$ получается почленным применением операции XOR(сложение по модулю 2) к исходным $n$ столбцам.
\end{definition}
\begin{theorem}
	Операция редукции не может сделать из запрещённой матрицы разрешённую
\end{theorem}
\begin{proof}
	Доказательство от противного. Пусть после редукции столбцов $a_1, a_2, \ldots a_n$ запрещённой матрицы $M_1$, получилась разрешённая $M_2$. Рассмотрим граф, в циклическое пространство которого вложена $IM_2$. Рассмотрим граф $G$, полученный из этого путём добавления $n - 1$ кратного ребра к ребру $a_1$ - эти рёбра будут соответствовать $a_2, a_3, \ldots a_n$ соотв. 
	
	Рассмотрим граф, состоящий только из рёбер какой-то строки. Он состоит из графа с двумя нечётными вершинами и, возможно, добавленными несколькими кратными ребрами. Операция XOR не изменяет чётность суммы переменных, а, значит, добавленные рёбра можно будет разбить на пары. Добавление пары кратных рёбер не изменяет чётности степени каждой из вершин. Следовательно, после добавления кратных рёбер, вершин с нечётной степенью будет также две.
	
	Таким образом, $IM_1$ вложено в $G$ - противоречие.
\end{proof}

Таким образом, запрещённые подпространства образуют идеал относительно редукций.

\begin{theorem}
	Для любой разрешённой матрицы $M$, или $IM$ вложено в пространство циклов простого (без кратных рёбер) полного графа, или после редукции каких-то двух столбцов $IM$ получается разрешённая матрица.
\end{theorem}
\begin{proof}
	Пусть $IM$ не вложено в пространство циклов какого-то простого полного графа, значит, оно вложено в пространство циклов таким образом, что какие-то два ребра кратны. В случае редукции столбцов, соответствующих этим двум рёбрам, получим разрешённый граф.
\end{proof}

Таким образом, можно сформулировать \textbf{Критерий запрещённой матрицы: матрица $M$ является запрещённой тогда и только тогда когда $IM$ не вложено в пространство какого-то простого полного цикла и редукция любых двух столбцов даёт запрещённое подпространство}.

Отсюда следует алгоритм поиска нижних единиц для запрещённых матриц для некоторого числа столбцов:

\begin{enumerate}
	\item Рекурсивно пробовать добавить к текущей матрице строку, которой ещё не было, до тех пор, пока матрица не станет запрещённой.
	\item Проверка того, что матрица запрещённая, осуществляется путём редукции всех возможных пар столбцов, а после рекурсивной проверки получившихся матриц на то, являются ли они разрешёнными. Если хотя бы одна разрешённая, то и искомая матрица разрешённая.
	\item Проверить, что матрица не вложена в пространство циклов какого-то простого полного графа.
	\item Проверить, что все матрицы, полученные удалением какой-то строки, разрешённые
	\item Если выполняются условия: матрица запрещённая, не вложена в простой полный граф и после вычитания любой строки получаем разрешённую, то вывести на экран. 
\end{enumerate}

Для скорости вычислений, используем ленивую динамику на шаге 2).

После выполнения программы, получаем верхние единицы для числа столбцов три и четыре(для удобства, случаи, получаемые друг из друга перестановкой столбцов, приводятся один раз):\\
$1 1 1\\
1 1 0\\
0 1 1\\
1 0 1$\\\\
$1 1 1 1\\
1 1 0 0\\
1 0 1 0\\
1 0 0 1\\
0 1 1 0\\
0 1 0 1\\
0 0 1 1$\\\\
$1 1 0 0\\
0 1 1 0\\
1 0 1 0\\
1 1 0 1\\
0 1 1 1\\
1 0 1 1$\\\\
$1 1 1 1\\
0 1 1 1\\
1 0 1 1\\
0 0 1 1\\
1 1 0 0$\\\\

На самом деле, если рассматривать не вложения матриц, а вложения подпространств, второй и третий случаи для 4х столбцов, содержат в себе нижнюю единицу для 3х столбцов.








\newpage

\section{Список источников}

\begin{thebibliography}{3}
	\bibitem{}
	J. A. Bondy, Small cycle double covers of graphs, in: Cycles and Rays (G. Hahn, G. Sabidussi, and R. Woodrow,
	eds.), NATO ASI Ser. C, Kluwer Academic Publishers, Dordrecht, 1990, 21-40.
	\bibitem{LiuLiu}
	F. Jaeger, A survey of the cycle double cover conjecture, Annals of Dis-
	crete Mathematics 27-Cycles in Graphs, North-Holland Mathematics Studies,
	27(1985), 1-12.
	\bibitem{Uitni}
	H. Whitney. Congruent graphs and the connectivity of graphs // Am. J. Math.. — 1932. — Т. 54. — С. 160-168.
\end{thebibliography}

\end{document}
